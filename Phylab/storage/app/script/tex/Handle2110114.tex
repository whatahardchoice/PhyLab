\subsection*{(1)反射式全息照相}
\begin{enumerate}
	\item 原始图像记录\\
	%插入图片
	可看到在感光板上有一个一元硬币的影像。
\end{enumerate}
\subsection*{(2)两次曝光测定铝板的杨氏模量}
\begin{enumerate}
	\item 原始图像记录\\
	%插入图片
	可看到在感光板上有干涉条纹的影像。
	\item 原始数据记录\\
	\begin{tabular}{|c|c|c|c|c|c|c|c|c|}
		\hline
		级次k & 1 & 2 & 3 & 4 & 5 & 6 & 7 & 8 \\
		\hline
		位置(cm) & %%d%% \\
		\hline
	\end{tabular}\\
	\begin{tabular}{|c|c|c|c|c|}
		\hline
		铝板长l(mm) & 铝板宽b(mm) & 铝板厚h(mm) & 激光波长(nm) & 砝码质量(g) \\
		\hline
		%%LEN_O%% & %%WID_O%% & %%THIC_O%% & %%LASER_O%% & %%WEIGHT_O%% \\
		\hline
	\end{tabular}
	\item 数据计算\\
	由原理可知$E= \displaystyle\frac{4F_yx^2(3L-x)}{(2k-1)\lambda bh^2}{(cos\alpha+cos\beta)}$。\\
	代入数据得\\
    \begin{equation}
    \begin{split}
    E&=\displaystyle\frac{8\times %%FY%%\times x^2(3\times %%LEN%% -x)}{(2k-1)%%LASER%%\times %%WID%% \times %%THIC%%^3} Pa\\
    &=\displaystyle\frac{ %%TEMP_RES_1%%\dot x^2 (%%LEN3%% -x)}{2k-1} Pa
    \end{split}
    \end{equation}
	则可得$x^2(%%LEN3%%-x)= \displaystyle\frac{E}{ %%TEMP_RES_1%% }{(2k-1)}$	(数值)\\
	令$Y = x^2(%%LEN3%%-x)$,$X= 2k-1$,则$Y=\displaystyle\frac{E}{ %%TEMP_RES_1%% }{X}$,令$Y = AX$\\
	知Y与X成线性关系,则可用一元线性回归法求解。\\
	由原始数据可求得X,Y,XY列表如下\\
	\begin{table}[htbp]
        \centering
        \small
        \setlength\tabcolsep{1pt}

		\begin{tabular}{|c|c|c|c|c|c|c|c|c|c|c|}
			\hline
			i & 1 & 2 & 3 & 4 & 5 & 6 & 7 & 8 & 总计 & 平均 \\
			\hline
			Xi & 1 & 3 & 5 & 7 & 9 & 11 & 13 & 15 & 64 & 8 \\
			\hline
			Yi & $%% d %%$  \\
			\hline
			XiYi & $%% d %%$  \\
			\hline
		\end{tabular}\\
	\end{table}
	由一元线性回归法可知:\\
	$A= \displaystyle\frac{\bar{X}\bar{Y}- \bar{XY}}{\bar{X}^2-\bar{X^2}}=%%A%%$\\
	即$\displaystyle\frac{E}{ %%TEMP_RES_1%% }=A=%%A%%$\\
	则$E={ %%TEMP_RES_1%% } \times %%A%%=%%E%%$
	\item 计算不确定度(不要求)\\
	$S(Y) = \sqrt{\displaystyle\frac{\sum (Y_i-Ax_i)^2}{k-2}} = %%SY%%$\\
	$S(A) = S(Y)\cdot\sqrt{\displaystyle\frac{1}{k(\bar{x^2}-\bar{x}^2)}} = %%SA%%$\\
	$u_a(A) = S(A) = %%SA%%$\\
	易知单位上难以确定,若仅以数值计算且忽略b类不确定度:\\
	即$u(A) = %%SA%%$\\
	则$u(E) =%%SA%%\times%%TEMP_RES_1%%=%%UE%%(GPa)$\\
	显然上述的计算过程是极其粗糙的。
	\item 最终结果表述\\
	$E±u(E) = (%%E_INT%% \pm %%UE%%)GPa$
	\item 相对误差计算\\
	$\eta =\displaystyle\frac{\left | E-E\text{理} \right |}{E\text{理}}\times 100\% = %%ETA%%$
	\item 误差分析\\
	①最主要的误差来源于记录数据时将第一级条纹所在处当成了坐标系原点而非装卡的位置。\\
	②条纹间距b,且因使影像,故测量时误差很大。\\
	③因视差等原因造成读数的误差。\\
	④由于长时间的使用。系统器件给定的值可能与实值有较大偏差。\\
\end{enumerate}