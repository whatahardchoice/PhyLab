\section*{五、数据处理}
        \subsection*{实验一:法拉第电解定律的验证}
    \subsubsection*{1、原始数据记录}
    \indent $T=||T||\circ C$\\
    \begin{tabular}{|c|c|c|c|c|c|}
        \hline
        输入电流 I(A) & 输入电压(V) & 时间t(s) & 电量It(c) & ${H}_{2}$产生量测定值(ml) & ${H}_{2}$产生量理论值(ml)\\
        %后文对应的建议的变量名称
        %输入电流为I[i] & 输入电压(V) & 时间为t[i] & 电量为It[i] & 产生量测定值(ml)V测[i] & 产生量理论值为VH[i]
        %输入电流,输入电压,时间均为测得的实验数据
        %电量和产生量理论值为利用实验数据所计算得值
        %产生量测定值也为实验测得数据,但似乎是一个实验固定值
        \hline
        ||I[0]|| & ||U1[0]|| & ||t[0]|| & ||It[0]|| & ||Vc[0]|| & ||VH[0]||\\
        %? & ? & ? & ? & ? & ?\\
        %I[0]&电压数据似乎不用&t[0]&It[0]&似乎为固定值&VH[0]
        \hline
        ||I[1]|| & ||U1[1]|| & ||t[1]|| & ||It[1]|| & ||Vc[1]|| & ||VH[1]||\\
        %? & ? & ? & ? & ? & ?\\
        %
        %I[1]&电压数据似乎不用&t[1]&It[1]&似乎为固定值&VH[1]
        %? & ? & ? & ? & ? & ?\\
        \hline
        ||I[2]|| & ||U1[2]|| & ||t[2]|| & ||It[2]|| & ||Vc[2]|| & ||VH[2]||\\
        %I[2]&电压数据似乎不用&t[2]&It[2]&似乎为固定值&VH[2]
        \hline
    \end{tabular}

    \subsubsection*{2、数据处理}
    \begin{enumerate}
        \item 电量计算\\
        因为$Q=It$

        所以${Q}_{1}=||I[0]||\times||t[0]|| = ||It[0]||c$。\\
        \indent 同理:${Q}_{2}=||I[1]||\times||t[1]|| = ||It[1]||c$
        ,${Q}_{3}=||I[2]||\times||t[2]|| = ||It[2]||c$

        \item ${H}_{2}$产生理论值的计算\\
        由公式${V}_{{H}_{2}}=\frac{273.16+T}{273.16}\times\frac{{p}_{0}}{p}\times\frac{It}{2F}\times22.4$

        $T=||T||\circ C$,${P}_{0}=P$

        得:${V}_{{H}_{21}}=\frac{273.16+||T||}{273.16}\times1\times\frac{||It[0]||}{2\times96500}\times22.4 = ||VH[0]||ml$

        ${V}_{{H}_{22}} = ||VH[1]||ml$

        ${V}_{{H}_{23}} = ||VH[2]||ml$

        \item 误差计算
        设相对误差为A

        所以${A}_{1}=\frac{{V}_{\text{测1}}-||VH[0]||}{||VH[0]||}\times100\% = ||A[0]||$

        ${A}_{2}=\frac{{V}_{\text{测2}}-||VH[1]||}{||VH[1]||}\times100\% = ||A[1]||$

        ${A}_{3}=\frac{{V}_{\text{测3}}-||VH[2]||}{||VH[2]||}\times100\% = ||A[2]||$\\

        结论:在误差允许范围内,法拉第电解定律成立

        \item 误差分析:\\

        1、由于电解池的效率不可能达到100\%,故测量值必然会小于理论值。是为本实验中的系统误差

        2、由于气水塔上的刻度只精确到毫升,故人眼读数会有一定的误差。
    \end{enumerate}

    \subsection*{实验二:燃料电池输出特性的测量}
        \subsubsection*{1、原始数据记录}
         \begin{table}[htbp]
                 \centering
                 \small
                 \setlength\tabcolsep{1pt}
                \begin{center}
                    \begin{tabular}{|c|c|c|c|c|c|c|c|c|c|c|c|}
                        \hline
                        输出电压U(V) & ||Ui|| \\
                        \hline
                        输出电流I(mA) & ||Ii|| \\
                        \hline
                        功率P=U*I(mW) & ||Pi|| \\
                        \hline
                    \end{tabular}
                \end{center}
         \end{table}

        \subsubsection*{2、数据处理}
            功率计算:P=UI,结果见上表

            $${p}_{max} = ||P_m|| mW$$

        \subsubsection*{3、燃料电池极化特性曲线}

            %插入图片
            \begin{figure}[H]
             \centering
              \includegraphics[width=13cm]{||pic1||.png}
            \end{figure}

    \subsection*{实验三:太阳能电池的特性测量}
        \subsubsection*{1、原始数据记录}
     \begin{table}[htbp]
                      \centering
                      \small
                      \setlength\tabcolsep{1pt}
        \begin{center}
            \begin{tabular}{|c|c|c|c|c|c|c|c|c|c|c|c|c|c|c|c|c|}
                \hline
                输出电压U(V)  & ||ui|| \\
                \hline
                输出电流I(A) & ||ii|| \\
                \hline
                功率P=U*I(W)  & ||pi|| \\
                \hline
            \end{tabular}
        \end{center}
        \end{table}

        短路电流$${I}_{sc}= ||I_sc|| A$$ 开路电压$${U}_{oc} = ||U_oc||V$$

        \subsubsection*{2、数据处理}

        P=U*I 结果参见上表。

        $${p}_{max} = ||p_m|| W$$

        所以填充因子$$FF=\frac{{U}_{m}{I}_{m}}{{U}_{OC}{I}_{SC}}=\frac{ ||p_m|| }{ ||I_sc|| \times ||U_oc|| } = ||FF||$$

        \subsubsection*{3、太阳能电池伏安特性曲线的绘制}
        %插入图片
         \begin{figure}[H]
            \centering
            \includegraphics[width=13cm]{||pic2||.png}
         \end{figure}

        $${U}_{m} = ||U_m|| V$$
        $${I}_{m} = ||I_m|| A$$

        \subsubsection*{4、该电池输出功率随输出电压的变化曲线}
        %插入图片
        \begin{figure}[H]
            \centering
            \includegraphics[width=13cm]{||pic3||.png}
        \end{figure}
