\documentclass[11pt,a4paper,oneside]{article}
\usepackage[UTF8,adobefonts]{ctex}

\usepackage{wrapfig}
\usepackage{indentfirst}
\usepackage{amsmath}
\usepackage{float}
\usepackage{ulem}

\usepackage[top=1in,bottom=1in,left=1.25in,right=1.25in]{geometry}

\usepackage{color}
\usepackage{xcolor}

\usepackage{multirow}
\usepackage{graphicx}
\usepackage{caption}
\usepackage{gensymb}
\usepackage{diagbox}

\begin{document}
\section*{五、数据处理}
$\rho=981kg/m^2$,\quad$g=9.792m/s^2$,\quad$\eta=1.83\times10^{-5}kg/m\cdot s$\\
\indent $b=6.17\times10^{-6}m\cdot cmHg=8.22\times10^{-3}m\cdot Pa$\\
\indent $P_{20}=76.0cmHg=1.0133\times10^5Pa$,\quad$d=5.00\times10^{-3} m$,\quad$e_0=1.6021773\times 10^{-19}c$\\
\indent 原始数据记录表\\
\indent \begin{tabular}{|c|c|c|c|c|c|c|c|}
\hline
 编号 & \diagbox{电压(V)}{时间(s)} &  &  &  &  &  & 平均\\
\hline
 1 & ||U[0]||  & ||t0[i]|| &  ||ave_t0|| \\
\hline
 2 & ||U[1]||  & ||t1[i]|| &  ||ave_t1|| \\
\hline
 3 & ||U[2]||  & ||t2[i]|| &  ||ave_t2|| \\
\hline
 4 & ||U[3]||  & ||t3[i]|| &  ||ave_t3|| \\
\hline
 5 & ||U[4]||  & ||t4[i]|| &  ||ave_t4|| \\
\hline
 6 & ||U[5]||  & ||t5[i]|| &  ||ave_t5|| \\
\hline
\end{tabular}\\
\indent $q=ne=\frac{0.9277\times10^{-14}}
{\left[t\left(1+2.264\times10^{-2}\sqrt{t}\right)\right]^\frac{3}{2}}
\cdot\frac{1}{V}$\\
代入计算\\
1组:\\
\indent $q_{1}=ne=\frac{0.9277\times10^{-14}}
{\left[||ave_t0||\left(1+2.264\times10^{-2}\sqrt{||ave_t0||}\right)\right]^\frac{3}{2}}
\cdot\frac{1}{||U[0]||}=||q0||c$\\
\indent $n_1=\frac{q_1}{e_0}\approx||n0||$,\quad$e_1=\frac{q_1}{n_1}=||e0||c$\\
\indent $\eta_1=||eta0||\%$\\
2组:\\
\indent $q_{2}=ne=\frac{0.9277\times10^{-14}}
{\left[||ave_t1||\left(1+2.264\times10^{-2}\sqrt{||ave_t1||}\right)\right]^\frac{3}{2}}
\cdot\frac{1}{||U[1]||}=||q1||c$\\
\indent $n_2=\frac{q_2}{e_0}\approx||n1||$,\quad$e_2=\frac{q_2}{n_2}=||e1||c$\\
\indent $\eta_2=||eta1||\%$\\
3组:\\
\indent $q_{3}=ne=\frac{0.9277\times10^{-14}}
{\left[||ave_t2||\left(1+2.264\times10^{-2}\sqrt{||ave_t2||}\right)\right]^\frac{3}{2}}
\cdot\frac{1}{||U[2]||}=||q2||c$\\
\indent $n_3=\frac{q_3}{e_0}\approx||n2||$,\quad$e_3=\frac{q_3}{n_3}=||e2||c$\\
\indent $\eta_3=||eta2||\%$\\
4组:\\
\indent $q_{4}=ne=\frac{0.9277\times10^{-14}}
{\left[||ave_t3||\left(1+2.264\times10^{-2}\sqrt{||ave_t3||}\right)\right]^\frac{3}{2}}
\cdot\frac{1}{||U[3]||}=||q3||c$\\
\indent $n_4=\frac{q_4}{e_0}\approx||n3||$,\quad$e_4=\frac{q_4}{n_4}=||e3||c$\\
\indent $\eta_4=||eta3||\%$\\
5组:\\
\indent $q_{5}=ne=\frac{0.9277\times10^{-14}}
{\left[||ave_t4||\left(1+2.264\times10^{-2}\sqrt{||ave_t4||}\right)\right]^\frac{3}{2}}
\cdot\frac{1}{||U[4]||}=||q4||c$\\
\indent $n_5=\frac{q_5}{e_0}\approx||n4||$,\quad$e_5=\frac{q_5}{n_5}=||e4||c$\\
\indent $\eta_5=||eta4||\%$\\
6组:\\
\indent $q_{6}=ne=\frac{0.9277\times10^{-14}}
{\left[||ave_t5||\left(1+2.264\times10^{-2}\sqrt{||ave_t5||}\right)\right]^\frac{3}{2}}
\cdot\frac{1}{||U[5]||}=||q5||c$\\
\indent $n_6=\frac{q_6}{e_0}\approx||n5||$,\quad$e_6=\frac{q_6}{n_6}=||e5||c$\\
\indent $\eta_6=||eta5||\%$\\
$\bar{e}=\frac{\sum_{i=1}^{n=6} e_i}{6}=||ave_e||c$,\quad$\eta=|\frac{\bar{e}-e_0}{e_0}|=||eta||$\\
不确定度计算:\\
\indent $U_a(e)=\sqrt{|\frac{\bar{e^2}-\bar{e}^2}{6-1}|}=||U_a||c$\\
\indent $e±U(e)=\left(||ave_e||±||U_a||\right)c$\\
%需要画表
\begin{figure}[H]
\centering
  \includegraphics[width=13cm]{||figurename||.png}
\end{figure}
\indent 由表可知,电荷约为$e_0=1.607\times10^{-19}c$整数倍,体现带电量的不连续性\\
\indent 元电荷$\bar{e}=\frac{\sum_{i=1}^{n} e_i}{n}=||ave_e||c$,\quad$\eta=|\frac{\bar{e}-e_0}{e_0}|=||eta||$
\section*{六、误差分析}
\begin{enumerate}
\item 由于油滴质量小,可能出现热运动,布朗运动引入误差。
\item 在时间测量上判断油滴是否达到测量距离的开始点和结束点时,人眼的主观读数误差,使测得时间有误差。
\item 油滴在上升或下降过程中因为温度,空气流动等原因挥发,质量稍有改变。
\end{enumerate}
\end{document}
