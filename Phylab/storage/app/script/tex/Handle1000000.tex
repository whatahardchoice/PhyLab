\subsection*{1.计算光栅常数d,并计算不确定度u(d)}
\subsubsection*{(1)原始数据记录表格}
\begin{center}
\begin{table}[htbp]
\begin{tabular}{|c|c|c|c|c|c|}
\hline
\multirow{2}*{\diagbox{测量次数}{测量级次}} &
\multicolumn{2}{c|}{$-1$级} & \multicolumn{2}{c|}{$+1$级} &
\multirow{2}*{$2{\theta}_1 = \displaystyle\frac{1}{2}\left[({\alpha}_1-{\beta}_1)-({\alpha}_2-{\beta}_2)\right]$}  \\
\cline{2-5}
& ${\alpha}_1$ & ${\beta}_1$ & ${\alpha}_2$ & ${\beta}_2$ & \\ \hline
1 & {{data[0][0]}} & {{data[1][0]}} & {{data[2][0]}} & {{data[3][0]}} & {{ave_d_theta1[0]}} \\ \hline
2 & {{data[0][1]}} & {{data[1][1]}} & {{data[2][1]}} & {{data[3][1]}} & {{ave_d_theta1[1]}}  \\ \hline
3 & {{data[0][2]}} & {{data[1][2]}} & {{data[2][2]}} & {{data[3][2]}} & {{ave_d_theta1[2]}}  \\ \hline
4 & {{data[0][3]}} & {{data[1][3]}} & {{data[2][3]}} & {{data[3][3]}} & {{ave_d_theta1[3]}}  \\ \hline
5 & {{data[0][4]}} & {{data[1][4]}} & {{data[2][4]}} & {{data[3][4]}} & {{ave_d_theta1[4]}} \\ \hline
\end{tabular}
\end{table}

\begin{table}[!hbp]
\begin{tabular}{|c|c|c|c|c|}
\hline
\multicolumn{2}{|c|}{$-2$级} & \multicolumn{2}{c|}{$+2$级} &
\multirow{2}*{$2{\theta}_2 = \frac{1}{2}\left[({\alpha}_1-{\beta}_1)-({\alpha}_2-{\beta}_2)\right]$}  \\
\cline{1-4}
${\alpha}_1$ & ${\beta}_1$ & ${\alpha}_2$ & ${\beta}_2$ & \\ \hline
{{data[4][0]}} &{{data[4][1]}} &{{data[4][2]}} &{{data[4][3]}} &{{d_theta2}} \\ \hline
\end{tabular}
\end{table}
\end{center}

\subsubsection*{(2)计算光栅常数d}

$$\overline{2{\theta}_1} = \displaystyle\frac{\sum_{k=1}^5 2{\theta}_1}{5} = {{ave_d_theta1}} $$
$$\overline{ {\theta}_1} = \displaystyle\frac12\ \overline{2{\theta}_1} = {{ave_theta1}} $$
$$\overline{ {\theta}_2} = \displaystyle\frac12\ \overline{2{\theta}_2} = {{theta2}}$$
$$\text{由\ }d\sin{\theta} = k{\lambda}\text{\ ,取\ }k = 1\text{\ 得\ }d = \frac{\lambda}{\sin{\theta}_1}$$
又钠黄光\ ${\lambda} = 589.3mm$
$$\therefore \ d = \displaystyle\frac{\lambda}{\sin{\theta}_1}={{d}}$$
$$\text{取\ }k=2 \displaystyle\text{得\ }d' = \frac{2\lambda}{\sin{\theta}_2}= {{d1}}$$
 
\subsubsection*{(3)计算不确定度u(d)}
\begin{enumerate}
  \item $\pm1$级d的不确定度
    $$u_a(\overline{2\theta}) = \displaystyle\sqrt{\frac{\sum_{i=1}^5{(2{\theta}_i-\overline{2{\theta}_1}})^2}{5\times4}}={{ua_ave_d_theta1}}$$
    $$u_b(\overline{2\theta}) = \displaystyle\frac{1}{\sqrt3} = {{ub_ave_d_theta1}}$$
    $$\text{不确定度合成为\ }u(\overline{2\theta}) = \sqrt{u_a^2(\overline{2\theta})+u_b^2(\overline{2\theta})} = {{u_ave_d_theta1}}$$
    $$u(\overline{ {\theta}_1})= \displaystyle\frac12\ u(\overline{2{\theta}_1}) = {{u_ave_theta1}}$$
    $$\text{由\ }d = \frac{\lambda}{\sin{ {\theta}_1}} \text{有\ } \ln d = \ln{\lambda}-\ln{\sin{ {\theta}_1} }$$
    $$\text{相对不确定度\ }\frac{u(d)}{d} = \displaystyle\sqrt{ {\left[\frac{\partial{\ln{\sin{ {\theta}_1}}}}{\partial{ {\theta}_1} }\ u({\theta}_1)\right]}^2} = \sqrt{ {\left[\frac{u({\theta}_1)}{\tan{ {\theta}_1}}\right]}^2} = {{u_d_d}}$$
    $$\therefore \ u(d) = \displaystyle d\ \frac{u(d)}{d} = {{u_d}}$$
  \item $\pm2$级d的不确定度
    $$\text{由\ }d' = \frac{\lambda}{\sin{ {\theta}_2}} \text{有\ } \ln d' = \ln{\lambda}-\ln{\sin{ {\theta}_2}}$$
    $$\text{而\ }u(2{\theta}_2) = u_b(2{\theta}_2) = \displaystyle\frac{1'}{\sqrt3} = {0.00962}^\circ = 0.000168rad $$
    $$\therefore u({\theta}_2) = \displaystyle\frac{1}{2}\ u(2{\theta}_2) = {0.00481}^{\circ} = 8.395\times 10^{-5}rad $$
    $$\therefore \text{相对不确定度\ }\frac{u(d')}{d'} = \displaystyle\sqrt{ {\left[\frac{\partial{\ln{\sin{ {\theta}_2}}}}{\partial{  {\theta}_2} }\ u({\theta}_2)\right]}^2} = \sqrt{ {\left[\frac{u({\theta}_2)}{\tan{ {\tan}_2}}\right]}^2} = {{u_d1_d1}}$$
    $$\therefore \ u(d') = \displaystyle d'\ \frac{u(d')}{d'} = {{u_d1}}$$
\end{enumerate}

\subsubsection*{(4)测量结果加权平均求d最佳值}
  测量结果: $$d \pm u(d) = {{d}} \pm {{u_d}}$$
  $$d' \pm u(d') = {{d1}} \pm {{u_d1}}$$
  $$\overline{d} = \displaystyle\frac{\frac{d}{u^2(d)}\ +\ \frac{d'}{u^2(d')}}{\frac{1}{u^2(d)}\ +\ \frac{1}{u^2(d')}} = {{ave_d}}$$
  $$u^2(\overline{d}) = \displaystyle\frac{1}{\frac{1}{u^2(d)}\ +\ \frac{1}{u^2(d')}} = {{u_ave_d2}}$$
  $$\therefore\ u(\overline{d}) = {{u_ave_d}}$$
  $$\therefore\text{光栅常数d的最终表达式为\ }\overline{d} \pm u(\overline{d}) $$

\subsection*{2.计算氢原子的里德伯常数$R_H + u(R_H)$;并通过加权平均获得$R_H$的最佳值$\overline{R_H} \pm u(\overline{R_H})$}
巴耳末系: $$ \displaystyle \frac{1}{\lambda} = R_H \ \left(\frac{1}{2^2}-\frac{1}{n^2}\right) (n = 3,4,5,6\dots) $$ 
当\ $n = 3$\ 时,光谱颜色为红光; 当\ $n = 5$\ 时,光谱颜色为蓝光; 当\ $n = 6$\ 时,光谱颜色为紫光; \\
以下将分别计算红光,蓝光,紫光对应的$R_H$:
\subsubsection*{(1)红光}
\begin{center}
\begin{table}[htbp]
\begin{tabular}{|c|c|c|c|c|c|}
\hline
\multirow{2}*{\diagbox{测量次数}{测量级次}} &
\multicolumn{2}{c|}{$-1$级} & \multicolumn{2}{c|}{$+1$级} &
\multirow{2}*{$2{\theta}_{\gamma} = \displaystyle\frac{1}{2}\left[({\alpha}_1-{\beta}_1)-({\alpha}_2-{\beta}_2)\right]$}  \\
\cline{2-5}
& ${\alpha}_1$ & ${\beta}_1$ & ${\alpha}_2$ & ${\beta}_2$ & \\ \hline
1 & {{data[5][0]}} & {{data[6][0]}} & {{data[7][0]}} & {{data[8][0]}} & {{ave_d_thetar[0]}} \\ \hline
2 & {{data[5][1]}} & {{data[6][1]}} & {{data[7][1]}} & {{data[8][1]}} & {{ave_d_thetar[1]}}  \\ \hline
3 & {{data[5][2]}} & {{data[6][2]}} & {{data[7][2]}} & {{data[8][2]}} & {{ave_d_thetar[2]}}  \\ \hline
4 & {{data[5][3]}} & {{data[6][3]}} & {{data[7][3]}} & {{data[8][3]}} & {{ave_d_thetar[3]}}  \\ \hline
5 & {{data[5][4]}} & {{data[6][4]}} & {{data[7][4]}} & {{data[8][04]}} &  {{ave_d_thetar[4]}} \\ \hline
\end{tabular}
\end{table}
\end{center}

\begin{enumerate}
  \item { }
      $$\overline{2{\theta}_{\gamma}} = \displaystyle\frac{\sum_{k=1}^5 2{\theta}_{\gamma}}{5} = {{ave_d_thetar}}$$
      $$\displaystyle\text{由\ }d\sin{\theta} = {\lambda}\text{\ 得\ }{\lambda}_{\gamma} = d\sin{\theta}_{\gamma} = d\sin{\frac{\overline{2{\theta}_{\gamma}}}{2}} = {{Bo_r}}$$
      $$\displaystyle\text{在巴耳末系中对应n取3,有\ }\frac{1}{\lambda}_{\gamma} = R_{H_1}\left(\frac{1}{2^2}-\frac{1}{3^2}\right)$$
      $$\therefore\ \displaystyle R_{H_{1}} = \frac{1}{ {\lambda}_{\gamma}}\left(\frac{1}{2^2}-\frac{1}{3^2}\right) = {{R_H1}}$$
  \item {不确定度的计算}
      $$u_a(\overline{2{\theta}_{\gamma}}) = \displaystyle\sqrt{\frac{\sum_{i=1}^{5} {(2{\theta}_{ {\gamma}_{i}}-\overline{2{\theta}_{\gamma]}}})^2}{5\times4}}={{ua_ave_d_thetar}}$$
      $$u_b(\overline{2\theta}) = \displaystyle\frac{1}{\sqrt3} = 9.6225\times10^{-3} = 1.679 \times 10^{-4} rad$$
      $$\therefore\text{不确定度合成为\ }u(\overline{2{\theta}_{\gamma}}) = \sqrt{u_a^2(\overline{2{\theta}_{\gamma}})+u_b^2(\overline{2{\theta}_{\gamma}})} = {{u_ave_d_thetar}}$$
      $$u(\overline{ {\theta}_{\gamma}})= \displaystyle\frac12\ u(\overline{2{\theta}_{\gamma}}) = {{u_ave_thetar}}$$
      $$\therefore{\theta}_{\gamma} \pm u({\theta}_{\gamma}) = {{ave_thetar}} \pm {{u_ave_thetar}}$$
      $$\text{而\ }\displaystyle R_{H_1} = \frac{1}{ {\lambda}_{\gamma}}\left(\frac{1}{2^2}-\frac{1}{3^2}\right) = \frac{7.2}{d\sin{\theta}_{\gamma}}$$
      $$\therefore\ln{R_{H_1}} = \ln{7.2} -\ln{d} - \ln{d\sin{\theta}_{\gamma}}$$
      $$\therefore\displaystyle \frac{u(R_{H_1})}{R_{H_1}} = \sqrt{ {\left[\frac{\partial{\ln{d}}}{\partial{d}}\ u(d)\right]}^2 + {\left[\frac{\partial{\ln{\sin{ {\theta}_{\gamma}}}}}{\partial{ {\theta}_{\gamma}}}\ u({\theta}_{\gamma})\right]}^2} = \sqrt{  {\left[\frac{u(d)}{d}\right]}^2 + {\left[\frac{u({\theta}_{\gamma})}{\tan{ {\theta}_{\gamma}}}\right]}^2} = {{u_R_H1_H1}}$$
      $$\therefore \ u(R_{H_1}) = \displaystyle R_{H_1}\ \frac{u(R_{H_1})}{R_{H_1}} = {{u_R_H1}}$$ 
      $$R_{H_1} \pm u(R_{H_1}) = {{R_H1}} \pm {{u_R_H1}}$$
\end{enumerate}

\subsubsection*{(2)蓝光(深绿)}
\begin{center}
\begin{table}[htbp]
\begin{tabular}{|c|c|c|c|c|c|}
\hline
\multirow{2}*{\diagbox{测量次数}{测量级次}} &
\multicolumn{2}{c|}{$-1$级} & \multicolumn{2}{c|}{$+1$级} &
\multirow{2}*{$2{\theta}_b = \displaystyle\frac{1}{2}\left[({\alpha}_1-{\beta}_1)-({\alpha}_2-{\beta}_2)\right]$}  \\
\cline{2-5}
& ${\alpha}_1$ & ${\beta}_1$ & ${\alpha}_2$ & ${\beta}_2$ & \\ \hline
1 & {{data[9][0]}} & {{data[10][0]}} & {{data[11][0]}} & {{data[12][0]}} & {{ave_d_thetab[0]}} \\ \hline
2 & {{data[9][1]}} & {{data[10][1]}} & {{data[11][1]}} & {{data[12][1]}} & {{ave_d_thetab[1]}}  \\ \hline
3 & {{data[9][2]}} & {{data[10][2]}} & {{data[11][2]}} & {{data[12][2]}} & {{ave_d_thetab[2]}}  \\ \hline
4 & {{data[9][3]}} & {{data[10][3]}} & {{data[11][3]}} & {{data[12][3]}} & {{ave_d_thetab[3]}}  \\ \hline
5 & {{data[9][4]}} & {{data[10][4]}} & {{data[11][4]}} & {{data[12][4]}} &  {{ave_d_thetab[4]}} \\ \hline 
\end{tabular}
\end{table}
\end{center}

\begin{enumerate}
  \item { }
      $$\overline{2{\theta}_b} = \displaystyle\frac{\sum_{k=1}^5 2{\theta}_b}{5} = {{ave_d_thetab}}$$
      $$\displaystyle\text{由\ }d\sin{\theta} = {\lambda}\text{\ 得\ }{\lambda}_b = d\sin{\theta}_b = d\sin{\frac{\overline{2{\theta}_b}}{2}} = {{Bo_b}}$$
      $$\displaystyle\text{在巴耳末系中对应n取4,有\ }\frac{1}{\lambda}_b = R_{H_2}\left(\frac{1}{2^2}-\frac{1}{4^2}\right)$$
      $$\therefore\ \displaystyle R_{H_2} = \frac{1}{ {\lambda}_b}\left(\frac{1}{2^2}-\frac{1}{4^2}\right) = {{R_H2}}$$
  \item {不确定度的计算}
      $$u_a(\overline{2{\theta}_b}) = \displaystyle\sqrt{\frac{\sum_{i=1}^5{(2{\theta}_{b_{i}}-\overline{2{\theta}_{\gamma]}}})^2}{5\times4}}={{ua_ave_d_thetab}}$$
      $$u_b(\overline{2\theta}) = \displaystyle\frac{1}{\sqrt3} = 9.6225\times10^{-3} = 1.679 \times 10^{-4} rad$$
      $$\therefore\text{不确定度合成为\ }u(\overline{2{\theta}_b}) = \sqrt{u_a^2(\overline{2{\theta}_b})+u_b^2(\overline{2{\theta}_b})} = {{u_ave_d_thetab}}$$
      $$u(\overline{ {\theta}_b})= \displaystyle\frac12\ u(\overline{2{\theta}_b}) = {{u_ave_thetab}}$$
      $$\therefore{\theta}_b \pm u({\theta}_b) = {{ave_thetab}} \pm {{u_ave_thetab}}$$
      $$\text{而\ }\displaystyle R_{H_2} = \frac{1}{ {\lambda}_b}\left(\frac{1}{2^2}-\frac{1}{4^2}\right) = \frac{5.333}{d\sin{\theta}_b}$$
      $$\therefore\ln{R_{H_2}} = \ln{5.333} -\ln{d} - \ln{d\sin{\theta}_b}$$
      $$\therefore\displaystyle \frac{u(R_{H_2})}{R_{H_2}} = \sqrt{ {\left[\frac{\partial{\ln{d}}}{\partial{d}}\ u(d)\right]}^2 + {\left[\frac{\partial{\ln{\sin{ {\theta}_b}}}}{\partial{ {\theta}_b}}\ u({\theta}_b)\right]}^2} = \sqrt{ {\left[\frac{u(d)}{d}\right]}^2 + {\left[\frac{u({\theta}_b)}{\tan{ {\theta}_b}}\right]}^2} = {{u_R_H2_H2}}$$
      $$\therefore \ u(R_{H_2}) = \displaystyle R_{H_2}\ \frac{u(R_{H_2})}{R_{H_2}} = {{u_R_H2}}$$ 
      $$R_{H_2} \pm u(R_{H_2}) = {{R_H2}} \pm {{u_R_H2}}$$
\end{enumerate}

\subsubsection*{(3)紫光(青)}
\begin{center}
\begin{table}[htbp]
\begin{tabular}{|c|c|c|c|c|c|}
\hline
\multirow{2}*{\diagbox{测量次数}{测量级次}} &
\multicolumn{2}{c|}{$-1$级} & \multicolumn{2}{c|}{$+1$级} &
\multirow{2}*{$2{\theta}_p = \displaystyle\frac{1}{2}\left[({\alpha}_1-{\beta}_1)-({\alpha}_2-{\beta}_2)\right]$}  \\
\cline{2-5}
& ${\alpha}_1$ & ${\beta}_1$ & ${\alpha}_2$ & ${\beta}_2$ & \\ \hline
1 & {{data[13][0]}} & {{data[14][0]}} & {{data[15][0]}} & {{data[16][0]}} & {{ave_d_thetap[0]}} \\ \hline
2 & {{data[13][1]}} & {{data[14][1]}} & {{data[15][1]}} & {{data[16][1]}} & {{ave_d_thetap[1]}}  \\ \hline
3 & {{data[13][2]}} & {{data[14][2]}} & {{data[15][2]}} & {{data[16][2]}} & {{ave_d_thetap[2]}}  \\ \hline
4 & {{data[13][3]}} & {{data[14][3]}} & {{data[15][3]}} & {{data[16][3]}} & {{ave_d_thetap[3]}}  \\ \hline
5 & {{data[13][4]}} & {{data[14][4]}} & {{data[15][4]}} & {{data[16][4]}} &  {{ave_d_thetap[4]}} \\ \hline 
\end{tabular}
\end{table}
\end{center}

\begin{enumerate}
  \item { }
      $$\overline{2{\theta}_p} = \displaystyle\frac{\sum_{k=1}^5 2{\theta}_p}{5} = {{ave_thetap}}$$
      $$\displaystyle\text{由\ }d\sin{\theta} = {\lambda}\text{\ 得\ }{\lambda}_p = d\sin{\theta}_p = d\sin{\frac{\overline{2{\theta}_p}}{2}} = {{Bo_p}}$$
      $$\displaystyle\text{在巴耳末系中对应n取5,有\ }\frac{1}{\lambda}_p = R_{H_3}\left(\frac{1}{2^2}-\frac{1}{5^2}\right)$$
      $$\therefore\ \displaystyle R_{H_3} = \frac{1}{ {\lambda}_p}\left(\frac{1}{2^2}-\frac{1}{5^2}\right) = {{R_H3}}$$
  \item {不确定度的计算}
      $$u_a(\overline{2{\theta}_p}) = \displaystyle\sqrt{\frac{\sum_{i=1}^5{(2{\theta}_{p_{i}}-\overline{2{\theta}_{\gamma]}}})^2}{5\times4}}={{ua_ave_d_thetap}}$$
      $$u_b(\overline{2\theta}) = \displaystyle\frac{1}{\sqrt3} = 9.6225\times10^{-3} = 1.679 \times 10^{-4} rad$$
      $$\therefore\text{不确定度合成为\ }u(\overline{2{\theta}_p}) = \sqrt{u_a^2(\overline{2{\theta}_p})+u_b^2(\overline{2{\theta}_p})} = {{u_ave_d_thetap}}$$
      $$u(\overline{ {\theta}_p})= \displaystyle\frac12\ u(\overline{2{\theta}_p}) = {{u_ave_thetap}}$$
      $$\therefore{\theta}_p \pm u({\theta}_p) = {{ave_thetap}} \pm {{u_ave_thetap}}$$
      $$\text{而\ }\displaystyle R_{H_3} = \frac{1}{ {\lambda}_p}\left(\frac{1}{2^2}-\frac{1}{5^2}\right) = \frac{1}{0.21}\ \frac{1}{d\sin{\theta}_p}$$
      $$\therefore\ln{R_{H_3}} = \ln{\frac{1}{0.21}} -\ln{d} - \ln{d\sin{\theta}_p}$$
      $$\therefore\displaystyle \frac{u(R_{H_3})}{R_{H_3}} = \sqrt{ {\left[\frac{\partial{\ln{d}}}{\partial{d}}\ u(d)\right]}^2 + {\left[\frac{\partial{\ln{\sin{ {\theta}_p}}}}{\partial{ {\theta}_p}}\ u({\theta}_p)\right]}^2} = \sqrt{ {\left[\frac{u(d)}{d}\right]}^2 + {\left[\frac{u({\theta}_p)}{\tan{ {\theta}_p}}\right]}^2} = {{u_R_H3_H3}}$$
      $$\therefore \ u(R_{H_3}) = \displaystyle R_{H_3}\ \frac{u(R_{H_3})}{R_{H_3}} = {{u_R_H3}}$$ 
      $$R_{H_3} \pm u(R_{H_3}) = {{R_H3}} \pm {{U_R_H3}}$$
\end{enumerate}

\subsection*{3.分别计算钠黄光k=1,2级的角散射率和分辨本领,并由此说明钠黄光双线能否被分开}
\subsubsection*{(1)色分辨本领}
$$\because\ N = \displaystyle\frac{D}{d} = {{N}}$$
$$\therefore\ R = \displaystyle\frac{\lambda}{ {\delta}_{\lambda}} = kN = \begin{cases} {{R1}} & k=1 \\ {{R2}} & k=2 \end{cases} $$
\subsubsection*{(2)角色散率}
由前面实验,$ \overline{ {\theta}_1} = {{ave_theta1}}, \overline{ {\theta}_1} = {{theta2}}$
由公式$D_{\theta} = \displaystyle\frac{k}{ds\sin{\theta}}$,求解可得\\
$$k=1\text{ \ 时,\ } D_{ {\theta}_1} = \displaystyle\frac{1}{d\sin{\overline{ {\theta}_1}}} = {{D_theta1}}$$
$$k=2\text{ \ 时,\ } D_{ {\theta}_2} = \displaystyle\frac{2}{d\sin{\overline{ {\theta}_2}}} = {{D_theta2}}$$
\subsubsection*{(3)钠黄光双线}
$${\theta}_1 = \arcsin{\frac{ {\lambda}_1}{d}} = {{L_theta1}}$$
$${\theta}_2 = \arcsin{\frac{ {\lambda}_2}{d}} = {{L_theta2}}$$
$$\Delta{\theta} = {\theta}_1 - {\theta}_2 = {{theta12}}$$
根据谱线的半角宽度计算公式可得
$${\delta}_{\theta} = \arcsin{\frac{2\lambda N_0}{Nd}} = {{theta0}}$$
$\because {\Delta}_{\theta} > {\delta}_{\theta} $ \\
$\therefore$本实验可将钠黄光的双线分开。