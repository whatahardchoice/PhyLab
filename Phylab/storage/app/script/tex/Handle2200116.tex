
\section*{五、数据处理}

\subsection*{1、定标}
在扫描周期0.2ms下,CCD波形一帧对应于示波器上的41个小格。因此,在0.1ms扫描周期下,每格对应的实际空间距离为$\Delta S=\frac{2700}{41*0.2}*11um=3.622mm$
\subsection*{2、布拉格衍射}
固定功率情况下,测量数据及当前频率(P为86mA)

\begin{tabular}{|c|c|c|c|c|c|c|c|c|}
	\hline 
	频率${f}_{S}/Hz$&& ||fsi||   \\ 
	\hline 
	偏转角$\phi/$div&& ||div1i||   \\ 
	\hline 
\end{tabular}
%${I}_{0}$ = 19格
由方程组
\left\{\begin{matrix}
$\frac{sin\theta}{siniB} = n$\\ 
$D = L\cdot 2\theta $
\end{matrix}\right.
由小角度原理可知:$iB = \frac{N\cdot\Delta{S}}{2Ln}$
由此可得偏转角$\varphi = 2iB = \frac{N\Delta{S}}{2Ln}$
测得 $L = {D}' = 0.568m$ , 并且已知$n = 2.386$
\begin{tabular}{|c|c|c|c|c|c|c|c|c|}
	\hline 
	频率${f}_{S}/Hz$&& ||fsi||   \\ 
	\hline 
	偏转角$\phi/$div&& ||div1i||   \\ 
    \hline 
	偏转角$\varphi/$rad&& ||rad1i||   \\ 
	\hline 
\end{tabular}
画出$\varphi - {f}_{s}$图并进行线性拟合
\begin{figure}[H]
\centering
  \includegraphics[width=13cm]{||pic1||.png}
\end{figure}
又由$\varphi = \frac{{\lambda}_{0}{n{V}_{s}}{f}_{s}$
$k = \frac{{\lambda}_{0}}{n{V}_{s}} = ||k1||$
计算出${V}_{s} = ||Vs||m/s$
理论值$\overline{{V}_{s}} = 3632m/s$
相对误差为$\eta = \left | \frac{{V}_{s}-\overline{{V}_{s}}}{\overline{{V}_{s}} \right | \times100\% = ||relative_err1|| $

\subsection*{3、声光调制}
\subsubsection*{(1)衍射光强度与超声波频率}
原始数据
\begin{tabular}{|c|c|c|c|c|c|c|c|c|}
	\hline 
	频率${f}_{s}/Hz$&& ||fsi||   \\ 
	\hline 
	光强格数$/$div&& ||div2i||   \\ 
	\hline 
\end{tabular}
绘制衍射光强度与超声度频率的关系曲线
\begin{figure}[H]
\centering
  \includegraphics[width=13cm]{||pic2||.png}
\end{figure}
可知,中心频率光强为$||div2_max||$格,对应大致频率为$||fs2_max||$MHz。

\subsubsection*{(3)、衍射光强与超声波功率}
原始数据($0.5$V/div)
\begin{tabular}{|c|c|c|c|c|c|c|c|c|c|c|c|c|}
	\hline 
	功率$P/mA$&& ||Pi||   \\ 
	\hline 
	光强$/$div&& ||Ii||   \\ 
	\hline 
\end{tabular}
绘制衍射光强度与超声波功率的关系曲线
\begin{figure}[H]
\centering
  \includegraphics[width=13cm]{||pic3||.png}
\end{figure}

计算最大衍射效率
把频率调为0 , ${I}_{0}$为$8.0$div
所以,$eta = \frac{{I}_{1}}{{I}_{0}} = \frac{8\times0.5}{8\times1} = ||eta2||\% $

\subsection*{4、喇曼-纳斯衍射}
(1)原始数据
两个衍射光家教距离为$||div3||$div
${I}_{0} = ||I0||$div , ${I}_{1} = ||I1||div$ ; ($P = 86$mA)
(2)计算衍射角
偏转角$\theta = \frac{N\cdot\Delta S}{L\cdot n} = ||theta1||$
理论值$\overline{\theta} = 0.00653$
相对误差$\eta = = \left | \frac{\theta-\overline{\theta}}{\overline{\heta} \right | \times100\% = ||eta2|| $
(2)最大衍射衍射效率
$\eta = \frac{{I}_{1}}{{I}_{0}}\times100\% = ||relative_err2||\% $