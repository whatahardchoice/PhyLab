%\section*{五、数据处理}
\subsection*{1、定标}
\indent 在扫描周期0.2ms下,CCD波形一帧对应于示波器上的41个小格。因此,在0.1ms扫描周期下,每格对应的实际空间距离为$\Delta S=\frac{2700}{41\times0.2} \times11um=3.622mm$
\subsection*{2、布拉格衍射}
\indent 固定功率情况下,测量数据及当前频率(P为86mA) \\
\begin{tabular}{|c|c|c|c|c|c|c|c|c|}
	\hline 
	频率${f}_{S}/Hz$& ||fsi||   \\ 
	\hline 
	偏转角$\phi/$div& ||div1i||   \\ 
	\hline 
\end{tabular} \\
%${I}_{0}$ = 19格
\indent 由方程组 \\
$\left\{\begin{matrix}
\frac{sin\theta}{siniB} = n\\ 
D = L\cdot 2\theta 
\end{matrix}\right. $
\indent 由小角度原理可知:$iB = \frac{N\cdot\Delta{S}}{2Ln}$ \\
\indent 由此可得偏转角$\varphi = 2iB = \frac{N\Delta{S}}{2Ln}$ \\
\indent 测得 $L = {D}' = 0.568m$ , 并且已知$n = 2.386$ \\
\begin{tabular}{|c|c|c|c|c|c|c|c|c|}
	\hline 
	频率${f}_{S}/Hz$& ||fsi||   \\ 
	\hline 
	偏转角$\phi/$div& ||div1i||   \\ 
    \hline 
	偏转角$\varphi/$rad& ||rad1i||   \\ 
	\hline 
\end{tabular}
\indent 画出$\varphi - {f}_{s}$图并进行线性拟合 \\
\begin{figure}[H]
\centering
  \includegraphics[width=13cm]{||pic1||.png}
\end{figure}
\indent 又由$\varphi = \frac{{\lambda}_{0}}{n{V}_{s}}{f}_{s}$ \\
\indent $k = \frac{{\lambda}_{0}}{n{V}_{s}} = ||k1||$ \\
\indent 计算出${V}_{s} = ||Vs||m/s$ \\
\indent 理论值$\overline{{V}_{s}} = 3632m/s$ \\
\indent 相对误差为$\eta = \left | \frac{{V}_{s}-\overline{{V}_{s}}}{\overline{{V}_{s}}} \right | \times100\% = ||relative_err1|| $
\subsection*{3、声光调制}
\subsubsection*{(1)衍射光强度与超声波频率}
\indent 原始数据 \\
\begin{tabular}{|c|c|c|c|c|c|c|c|c|}
	\hline 
	频率${f}_{s}/Hz$& ||fsi||   \\ 
	\hline 
	光强格数$/$div& ||div2i||   \\ 
	\hline 
\end{tabular} \\
\indent 绘制衍射光强度与超声度频率的关系曲线 \\
\begin{figure}[H]
\centering
  \includegraphics[width=13cm]{||pic2||.png}
\end{figure}
\indent 可知,中心频率光强为$||div2_max||$格,对应大致频率为$||fs2_max||$MHz。 
\subsubsection*{(2)衍射光强与超声波功率}
\indent 原始数据($0.5$V/div) \\
\begin{tabular}{|c|c|c|c|c|c|c|c|c|c|c|c|c|}
	\hline 
	功率$P/mA$& ||Pi||   \\ 
	\hline 
	光强$/$div& ||Ii||   \\ 
	\hline 
\end{tabular}
\indent 绘制衍射光强度与超声波功率的关系曲线 \\
\begin{figure}[H]
\centering
  \includegraphics[width=13cm]{||pic3||.png}
\end{figure}
\indent 计算最大衍射效率 \\
\indent 把频率调为0 , ${I}_{0}$为$8.0$div \\
\indent 所以,$eta = \frac{{I}_{1}}{{I}_{0}} = \frac{8\times0.5}{8\times1} = ||eta2||\% $ 
\subsection*{4、喇曼-纳斯衍射}
\indent (1)原始数据 \\
\indent 两个衍射光家教距离为$||div3||$div \\
\indent ${I}_{0} = ||I0||$div , ${I}_{1} = ||I1||div$ ; ($P = 86$mA) \\
\indent (2)计算衍射角 \\
\indent 偏转角$\theta = \frac{N\cdot\Delta S}{L\cdot n} = ||theta1||$ \\
\indent 理论值$\overline{\theta} = 0.00653$ \\
\indent 相对误差$\eta = = \left | \frac{\theta-\overline{\theta}}{\overline{\theta}} \right | \times100\% = ||eta2|| $ \\
\indent (2)最大衍射衍射效率 \\
\indent $\eta = \frac{{I}_{1}}{{I}_{0}}\times100\% = ||relative_err2||\% $ \\